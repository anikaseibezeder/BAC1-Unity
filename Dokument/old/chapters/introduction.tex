\chapter{Introduction}
\label{cha:Introduction}

Nowadays creating realistic 3D games is getting a lot easier, due to various tools like Unity3D or the Unreal Engine. Game developers need little to no programming skills to make games in a minimum amount of time, because a great amount of premade assets and scripts are available for a very low cost. Most of the developing work can be done in the graphical editor, like creating 3D figures and adding properties, which can include self made scripts or physical properties. The majority of the existing game editors provide developing on multiple platforms including Smartphones and the most common game consoles. However Unity3D is better suited for devices with low end graphics cards, because it is more lightweight than the Unreal Engine. Compared to it, the available assets of Unity3D are not as high quality, but more economical on memory and graphics. \\ \\
The main points of this paper are: outlining the workflow of the game development process in Unity3D which consists of creating an environment, making a Third-Person-Character and some enemies, and comparing the basic functionality and simplicity of Unity3D and the Unreal Engine. The advantages and disadvantages of both editors are examined and demonstrated. \\
The first point of the game development process in Unity3D is to create an environment, with a small forest, some mountains and a few cottages. Then some textures are added to the components.
Secondly the Third-Person-Character is linked with a  script, so it can move around in the environment.
The third point is to create some enemies, that are chasing the player. 
Lastly animations for both, the character and the enemies, are provided and added to the game. \\ \\
The goal of this paper is to demonstrate the simplicity of Unity3D by creating a small 3D game in Unity3D.To keep the costs low, only freely available assets are used. The focus is not on scripting, but on the features of the graphical editor provided by Unity3D and the Unreal Engine. 

